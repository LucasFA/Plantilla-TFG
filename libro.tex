% !TEX program = pdflatex
% !TEX encoding = UTF-8 Unicode

% Plantilla, basada en la clase `scrbook` del paquete KOMA-script,  para la elaboración de un TFG siguiendo las directrices del la comisión del Grado en Matemáticas de la Universidad de Granada.

% Francisco Torralbo Torralbo

\documentclass[print, color]{ugrTFG}

% VERSIÓN ELECTRÓNICA PARA TABLETA
% Cambiando la opción "print" por "tablet" generaremos un pdf adaptado para leerlo en tabletas de 9 pulgadas.

% -------------------------------------------------------------------
% INFORMACIÓN DEL TFG Y EL AUTOR
% -------------------------------------------------------------------

\newcommand{\miTitulo}{Título del trabajo\xspace}
\newcommand{\miNombre}{Nombre apellidos\xspace}
\newcommand{\miGrado}{Grado en Matemáticas}
\newcommand{\miFacultad}{Facultad de Ciencias}
\newcommand{\miUniversidad}{Universidad de Granada}

% Añadir tantos tutores como sea necesario separando cada uno de ellos mediante el comando `\medskip` y una línea en blanco
\newcommand{\miTutor}{
  Nombre del tutor 1 \\ \emph{Departamento del tutor 1} 

  % Añadir tantos tutores como sea necesario. 

  \medskip
  Nombre del tutor 2 \\ \emph{Departamento del tutor 2}
}
\newcommand{\miCurso}{2023-2024\xspace}

\hypersetup{
	pdftitle={\miTitulo},
	pdfauthor={\textcopyright\ \miNombre, \miFacultad, \miUniversidad}
}

\begin{document}

\maketitle

% -------------------------------------------------------------------
% FRONTMATTER
% -------------------------------------------------------------------
\frontmatter % Desactiva la numeración de capítulos y usa numeración romana para las páginas

% !TeX root = ../libro.tex
% !TeX encoding = utf8
%
%*******************************************************
% Declaración de originalidad
%*******************************************************

\thispagestyle{empty}

\hfill\vfill

\textsc{Declaración de originalidad}\\\bigskip

D./Dña. \miNombre \\\medskip

Declaro explícitamente que el trabajo presentado como Trabajo de Fin de Grado (TFG), correspondiente al curso académico \miCurso, es original, entendida esta, en el sentido de que no ha utilizado para la elaboración del trabajo fuentes sin citarlas debidamente.
\medskip

En Granada a \today 
\begin{flushright} 
Fdo: \miNombre 

\end{flushright}

\vfill

\cleardoublepage
\endinput
   
% !TeX root = ../tfg.tex
% !TeX encoding = utf8

%*******************************************************
% Dedication
%*******************************************************
\thispagestyle{empty}
\phantomsection 
\pdfbookmark[1]{Dedicatoria}{Dedicatoria}

\hfill
\vfill

\begin{flushright}
\itshape
Dedicatoria (opcional) \\
Ver archivo \texttt{preliminares/dedicatoria.tex}
\end{flushright}

\vfill

\cleardoublepage
\endinput
                % Opcional
% !TeX root = ../libro.tex
% !TeX encoding = utf8

%*******************************************************
% Table of Contents
%*******************************************************
\phantomsection
\pdfbookmark[0]{\contentsname}{toc}

\setcounter{tocdepth}{2} % <-- 2 includes up to subsections in the ToC
\setcounter{secnumdepth}{3} % <-- 3 numbers up to subsubsections

% \manualmark
% \markboth{\textsc{\contentsname}}{\textsc{\contentsname}}
\tableofcontents 

%*******************************************************
% List of Figures and of the Tables
%*******************************************************

    % *******************************************************
    %  List of Figures
    % *******************************************************    
    \phantomsection 
    \listoffigures

    %*******************************************************
    % List of Tables
    %*******************************************************
    \phantomsection 
    \listoftables
    
    %*******************************************************
    % List of Listings
    % The package \usepackage{listings} is needed
    %*******************************************************      
	  % \phantomsection 
    % \renewcommand{\lstlistlistingname}{Listados de código}
    % \lstlistoflistings 

\cleardoublepage
            
% !TeX root = ../libro.tex
% !TeX encoding = utf8

%*******************************************************
% Agradecimientos
%*******************************************************

\chapter{Agradecimientos}

Agradecimientos del libro.

\cleardoublepage
\endinput
            % Opcional

% !TeX root = ../libro.tex
% !TeX encoding = utf8
%
%*******************************************************
% Summary
%*******************************************************


\chapter{Summary}

An english summary of the project (around 800 and 1500 words are recommended).

\endinput
                    
% !TeX root = ../tfg.tex
% !TeX encoding = utf8
%
%*******************************************************
% Introducción
%*******************************************************

% \manualmark
% \markboth{\textsc{Introducción}}{\textsc{Introducción}} 

\chapter{Introducción}

De acuerdo con la comisión de grado, el TFG debe incluir una introducción en la que se describan claramente los objetivos previstos inicialmente en la propuesta de TFG, indicando si han sido o no alcanzados, los antecedentes importantes para el desarrollo, los resultados obtenidos, en su caso y las principales fuentes consultadas.

Ver archivo \texttt{preliminares/introduccion.tex}

\endinput
               

% -------------------------------------------------------------------
% MAINMATTER
% -------------------------------------------------------------------
\mainmatter % activa la numeración de capítulos, resetea la numeración de las páginas y usa números arábigos

\part{Primera parte} % Dividir un libro en partes OPCIONAL

% !TeX root = ../libro.tex
% !TeX encoding = utf8

\chapter{Primer capítulo}\label{ch:primer-capitulo}

\section{Primera sección}
Cita de libro \cite{Euler1982, Euler1984, Euler1985}, \cite{EulerWiki} recurso online

\index{Leonard!Euler|textbf}

Lorem ipsum dolor sit amet\marginline{\footnotesize Esto es una nota al margen}, \textsf{consetetur sadipscing elitr}, sed diam nonumy eirmod
tempor invidunt ut labore et \texttt{dolore magna aliquyam erat}, sed diam voluptua. At vero eos et accusam et justo duo dolores et ea rebum. Stet clita kasd gubergren, no sea takimata sanctus est Lorem ipsum dolor sit amet.\index{Matemático|see{Leonard, Euler}}
\[
F(x) = \int_a^b f(t)\, \mathrm{d}x
\]


\subsection{Subsección}

{\ttfamily
\textbf{Lorem ipsum dolor sit amet}, consetetur sadipscing elitr, sed diam nonumy eirmod tempor invidunt ut \textit{labore et dolore magna aliquyam erat}, sed diam voluptua. At vero eos et accusam et justo duo dolores et ea rebum. Stet clita kasd gubergren, no sea takimata sanctus est Lorem ipsum dolor sit amet.
}

\newpage
\section{Primera sección}
Cita de libro \cite{Euler1982, Euler1984, Euler1985}, \cite{EulerWiki} recurso online

\index{Leonard!Euler|textbf}

Lorem ipsum dolor sit amet\marginline{\footnotesize Esto es una nota al margen}, \textsf{consetetur sadipscing elitr}, sed diam nonumy eirmod
tempor invidunt ut labore et \texttt{dolore magna aliquyam erat}, sed diam voluptua. At vero eos et accusam et justo duo dolores et ea rebum. Stet clita kasd gubergren, no sea takimata sanctus est Lorem ipsum dolor sit amet.\index{Matemático|see{Leonard, Euler}}
\[
  F(x) = \int_a^b f(t)\, \mathrm{d}x
\]


\subsection{Subsección}

{\ttfamily
  \textbf{Lorem ipsum dolor sit amet}, consetetur sadipscing elitr, sed diam nonumy eirmod tempor invidunt ut \textit{labore et dolore magna aliquyam erat}, sed diam voluptua. At vero eos et accusam et justo duo dolores et ea rebum. Stet clita kasd gubergren, no sea takimata sanctus est Lorem ipsum dolor sit amet.
}

\newpage
Lorem ipsum dolor sit amet, consetetur sadipscing elitr, sed diam nonumy eirmod
tempor invidunt ut labore et dolore magna aliquyam erat, sed diam voluptua. At
vero eos et accusam et justo duo dolores et ea rebum. Stet clita kasd gubergren,
no sea takimata sanctus est Lorem ipsum dolor sit amet.

\newpage
Lorem ipsum dolor sit amet, consetetur sadipscing elitr, sed diam nonumy eirmod
tempor invidunt ut labore et dolore magna aliquyam erat, sed diam voluptua. At
vero eos et accusam et justo duo dolores et ea rebum. Stet clita kasd gubergren,
no sea takimata sanctus est Lorem ipsum dolor sit amet.
\endinput
%------------------------------------------------------------------------------------
% FIN DEL CAPÍTULO. 
%------------------------------------------------------------------------------------

% Añadir tantos capítulos como sea necesario

\cleardoublepage\part{Segunda parte}
% !TeX root = ../libro.tex
% !TeX encoding = utf8

\chapter{Segundo capítulo}
\dictum[Leonard Euler]{Mathematicians have tried in vain to this day to discover some order in the sequence of prime numbers, and we have reasons to believe that it is a mystery into which the human mind will never penetrate.}

\section{Primera sección}

\endinput
%------------------------------------------------------------------------------------
% FIN DEL CAPÍTULO. 
%------------------------------------------------------------------------------------


% -------------------------------------------------------------------
% APPENDIX: Opcional
% -------------------------------------------------------------------

\appendix % Reinicia la numeración de los capítulos y usa letras para numerarlos
\pdfbookmark[-1]{Apéndices}{appendix} % Alternativamente podemos agrupar los apéndices con un nuevo \part{Apéndices}

% !TeX root = ../libro.tex
% !TeX encoding = utf8

\chapter{Primer apéndice}\label{ap:apendice1}

Lorem ipsum dolor sit amet, consetetur sadipscing elitr, sed diam nonumy eirmod
tempor invidunt ut labore et dolore magna aliquyam erat, sed diam voluptua. At
vero eos et accusam et justo duo dolores et ea rebum. Stet clita kasd gubergren,
no sea takimata sanctus est Lorem ipsum dolor sit amet.

\endinput
%------------------------------------------------------------------------------------
% FIN DEL APÉNDICE. 
%------------------------------------------------------------------------------------

% Añadir tantos apéndices como sea necesario 

% -------------------------------------------------------------------
% GLOSARIO: Opcional
% -------------------------------------------------------------------

% !TeX root = ../libro.tex
% !TeX encoding = utf8

\chapter*{Glosario}
\addcontentsline{toc}{chapter}{Glosario} % Añade el glosario a la tabla de contenidos

La inclusión de un glosario es opcional.

Archivo: \texttt{glosario.tex}

\begin{description} 
  \item[$\mathbb{R}$] Conjunto de números reales.

  \item[$\mathbb{C}$] Conjunto de números complejos.

  \item[$\mathbb{Z}$] Conjunto de números enteros.
\end{description}
\endinput
 

% -------------------------------------------------------------------
% BACKMATTER
% -------------------------------------------------------------------

\backmatter % Desactiva la numeración de los capítulos
\pdfbookmark[-1]{Referencias}{BM-Referencias}

% BIBLIOGRAFÍA
%-------------------------------------------------------------------

\setbibpreamble{Las referencias se listan por orden alfabético. Aquellas referencias con más de un autor están ordenadas de acuerdo con el primer autor.\par\bigskip}
\bibliographystyle{alpha} 
\begin{small} 
  \bibliography{library.bib}
\end{small}


\end{document}
